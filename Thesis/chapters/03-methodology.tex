\chapter{Methodology}

\section{Overview of Research Approach}
\begin{itemize}
    \item Two-stage methodology: numerical simulation and physical testing.
    \item Aim: identify key nonlinear dynamic behaviours using Gaussian Process Regression (GPR) and Bayesian Optimisation (BO) within a control-based testing framework.
    \item Rationale for selecting this approach over conventional uniform sampling methods.
\end{itemize}

\section{Simulation Stage}
\begin{itemize}
    \item Development of a nonlinear beam model (Duffing-type nonlinearity) in MATLAB.
    \item Simulation of steady-state response to harmonic forcing.
    \item Inclusion of artificial noise to emulate measurement uncertainty.
    \item Implementation of BO + GPR for adaptive frequency/amplitude test point selection.
    \item Simulation workflow and data processing steps.
\end{itemize}

\section{Physical Testing Stage}
\begin{itemize}
    \item Initial plan: testing on a Rolls-Royce rotor blade test rig.
    \item Reasons for switching to simplified beam-spring (thin ruler) setup.
    \item Description of experimental setup:
    \begin{itemize}
        \item Geometry and boundary conditions.
        \item Instrumentation: accelerometers, force sensors, data acquisition system.
        \item Excitation method and control hardware.
    \end{itemize}
    \item Control-based testing methodology for stabilising and measuring responses.
\end{itemize}

\section{Control-Based Testing Implementation}
\begin{itemize}
    \item Control strategy for maintaining desired frequency/amplitude.
    \item Feedback control loop design.
    \item Modifications required for testing simplified setup.
\end{itemize}

\section{Data Processing and Modelling}
\begin{itemize}
    \item Frequency Response Function (FRF) estimation from force and displacement/acceleration measurements.
    \item Training of GPR models on collected datasets.
    \item Use of BO to select subsequent test points based on model uncertainty.
    \item Stopping criteria for experimental campaigns.
\end{itemize}

\section{Evaluation Metrics}
\begin{itemize}
    \item Prediction accuracy compared to uniform sampling baseline.
    \item Reduction in number of test points required.
    \item Time efficiency compared to traditional methods.
\end{itemize}

\section{Summary}
\begin{itemize}
    \item Recap of methodology and its alignment with project objectives.
    \item Transition to results and analysis chapter.
\end{itemize}
