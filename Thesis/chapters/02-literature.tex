\chapter{Literature Review}

\section{Nonlinear Dynamics in Aerospace Structures}
\begin{itemize}
    \item Definition and classification of nonlinear behaviours (bifurcations, jump phenomena, amplitude-dependence).
    \item Common sources of nonlinearity: geometry, boundary conditions, material properties.
    \item Relevance to aerospace components (rotor blades, panels, turbine blades).
    \item Challenges in predicting and measuring nonlinear responses.
\end{itemize}

\section{Experimental Approaches to Characterising Nonlinear Dynamics}
\begin{itemize}
    \item Stepped-sine and swept-sine methods.
    \item Random excitation and FRF estimation.
    \item Experimental modal analysis for nonlinear systems.
    \item Limitations of conventional methods for unstable or high-amplitude states.
\end{itemize}

\section{Continuation Methods}
\begin{itemize}
    \item Numerical continuation: principles and applications in structural dynamics.
    \item Bifurcation tracking and backbone curve identification.
    \item Experimental continuation: challenges and benefits.
    \item Control-Based Continuation (CBC):
    \begin{itemize}
        \item Methodology and control requirements.
        \item Applications to physical systems (beams, rotors, aerospace-relevant structures).
        \item Limitations in high-noise or complex MDOF systems.
    \end{itemize}
\end{itemize}

\section{Data-Driven Surrogate Modelling}
\begin{itemize}
    \item Overview of surrogate modelling for dynamic systems.
    \item Gaussian Process Regression (GPR):
    \begin{itemize}
        \item Fundamentals and probabilistic nature (mean + uncertainty).
        \item Use in mechanical and aerospace testing.
        \item Strengths and limitations compared to other models.
    \end{itemize}
\end{itemize}

\section{Efficient Experimental Design}
\begin{itemize}
    \item Concept of Optimal Experimental Design (OED).
    \item Bayesian Optimisation (BO):
    \begin{itemize}
        \item Principles and acquisition functions.
        \item Applications in engineering system testing.
    \end{itemize}
    \item Previous work combining BO and GPR for system identification.
\end{itemize}

\section{Integration of CBC, GPR, and BO}
\begin{itemize}
    \item Synergies between control-based testing and adaptive sampling.
    \item Previous frameworks combining physical testing with data-driven optimisation.
    \item Identified gaps:
    \begin{itemize}
        \item Limited application to physical nonlinear structures in aerospace context.
        \item Lack of validation beyond simulations.
    \end{itemize}
\end{itemize}

\section{Summary of Literature and Research Gap}
\begin{itemize}
    \item Consolidate findings from Sections~\ref{sec:Nonlinear}--\ref{sec:Integration}.
    \item Highlight where existing approaches fall short.
    \item State the unique contribution of this work in addressing these gaps.
\end{itemize}
