\chapter{Introduction}

\section{Background \& Motivation}
In Aerospace Applications Rotor-dynamics regularly exhibit complex nonlinear behaviors, such as bifurcations, multiple solution branches, and amplitude-dependent dynamics. Nonlinearities in the dynamics can arise from various sources which can be categorized into 3 natures, geometrical, boundary conditions or material. A rotorblade in a turbine is often subjected to all three sources, geometrical due to its unqiue shape, boundry conditon: arizing from its contact mechanism with the central rotor, and finally material nonlinearity due to intense heat and vibration leading to either hardening or softening behavior
\cite{akayContinuationAnalysisNonlinear2021a, chipatoEffectGravityinducedAsymmetry2018, varneyNonlinearPhenomenaBifurcations2015}. understanding the nonlinear dynamics of a rotorblade in operation is critical for safety and optimal performance.

Tools like numerical continuation and bifurcation analysuis have enabled 
\cite{PDFNonlinearDynamics2025}

Nonlinear dynamic behavior is common in aerospace structures, especially in rotor blades subject to large-amplitude vibrations.
Traditional test methods can be time-consuming and may fail to capture unstable or high-amplitude responses.
There is a need for efficient, targeted strategies to capture nonlinear behavior with minimal test points.
Aerospace manufacturers seek methods that reduce testing costs while maintaining accuracy.

\section{Problem Statement}
Current test approaches are often uniformly sampled, leading to wasted time in regions of low dynamic interest.
Capturing unstable dynamic states is challenging without specialized control methods.
There is a lack of integrated frameworks combining data-driven models and optimal test selection.

\section{Objectives}
Develop a methodology using Bayesian Optimization (BO) and Gaussian Process Regression (GPR).
Validate in two stages: (i) numerical simulation of a nonlinear beam model; (ii) physical testing of a simplified setup.
Demonstrate improved efficiency over uniform sampling.

\section{Thesis Structure}
Chapter~1 introduces the problem, objectives, and scope.
Chapter~2 reviews literature.
Chapter~3 details methods.
Chapter~4 presents results.
Chapter~5 concludes and outlines future work.
